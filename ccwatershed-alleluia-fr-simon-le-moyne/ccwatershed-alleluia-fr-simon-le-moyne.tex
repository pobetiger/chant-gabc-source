% !TEX TS-program = lualatex
% !TEX encoding = UTF-8

% This is a simple template for a LuaLaTeX document using gregorio scores.

%\documentclass[11pt]{article} % use larger type; default would be 10pt
\documentclass[11pt]{book} % use larger type; default would be 10pt

% usual packages loading:
\usepackage{luatextra}
\usepackage{graphicx} % support the \includegraphics command and options
\usepackage{geometry} % See geometry.pdf to learn the layout options. There are lots.
%\geometry{a4paper} % or letterpaper (US) or a5paper or....
\geometry{letterpaper}
\usepackage{gregoriotex} % for gregorio score inclusion
\usepackage{fullpage} % to reduce the margins

% choose the language of the document here
\usepackage[latin]{babel}

% use the two following package for using normal TeX fonts
\usepackage[T1]{fontenc}
\usepackage[utf8]{luainputenc}

% If you use usual TeX fonts, here is a starting point:
%\usepackage{times}
\usepackage{newcent}
%\usepackage{palatino}
%\usepackage{pslatex}

% to change the font to something better, you can install the kpfonts package (if not already installed). To do so
% go open the "TeX Live Manager" in the Menu Start->All Programs->TeX Live 2010

% here we begin the document
\begin{document}

% Here we set the space around the initial.
% Please report to http://home.gna.org/gregorio/gregoriotex/details for more details and options
\setspaceafterinitial{2.2mm plus 0em minus 0em}
\setspacebeforeinitial{2.2mm plus 0em minus 0em}

% Here we set the initial font. Change 43 if you want a bigger initial.
\def\greinitialformat#1{%
  {\fontsize{43}{43}\selectfont #1}%
}

% We set red lines here, comment it if you want black ones.
\redlines


\frontmatter

%%
% Front Cover
%%
\title{ALLELUIA IN HONOR OF FR. SIMON LE MOYNE}
\author{Ben Sung Hsu\\
  \texttt{pobetiger(at)gmail.com}}
\date{\today}
\maketitle

%%
% Inside Cover
%%

\begin{paragraph}\noindent
Some information about the book goes here.
\end{paragraph}


%%
% Preface
%%

\chapter{PREFACE}
\begin{paragraph}
In the year 2009, I started working at my first job in Dalals, Texas. 
And I also returned to my home parish at Sacred Heart of Jesus Chinese
Parish in Plano, Texas. I started to sing at the choir and played the guitar
for the English Mass. And soon I took over the music planner's position.
My time away at college had provided me with some pratical skills in the planning
and preparation for providing music at Mass. But it had not been a great experience
of catechesis. \\

(Greatly shortening the story) \\

Through many years of studying and reading, I have gradually come
to an understanding of the Church askes of us musicians to sing at the Mass.
Out of these many years came the conclusion that as a musical community, we must
follow the instructions of the GIRM and the spirit of the Second Vatican Council
 to learn more about Gregorian Chant. In that light, various google and 
youtube searches were performed. Corpus Christi Watershed, Musica Sacra, and 
the Church Musician Association of America was found. And from the Corpus Christi
Watershed website came this particular piece of Alleluia dedicated in honor of
Fr. Simon Le Moyne. After listening to a couple of them, this piece proved not
too difficult to learn and implement especially with the requirement to sing
the correct verse proper to the feast. In addition, it came with a piano
 accompliment that assist our young cantors to sing and is not too
 difficult or overpowering. With some practice, our choir started using
this Gospel Acclamation. This is the first time in our parish that the
correct accompanying Alleluia Verse was used. There were times when the parishioners
questioned about the change. But after some time, it became expected.\\

When we were traveling to other parishes, and they would be singing from some
seasonal or collection of verses in some hymnal from the 70's, my finacee looked at
me and asked what happend to the Verses while point at it in the missalette she
was looking at. I smiled and thank God for Corpus Christi Watershed and the CMAA
in supporting and assisting us in our liturgical endeavors to sing the Mass.

This collection is simply my rendition of all the verses so that I don't have to
go to the website every week to download the PDF. I guess there's a point when 
a man learns to fish, he attempts to open a restaurant along the seashore.

\begin{center}Ben Sung Thomas More Hsu\\\today
\end{center}

\end{paragraph}


%%
% Dedication page
%%

\chapter{DEDICATION}
\begin{paragraph}\\
Dedicated Fr. Edward Kokarchik,\\
who chanted the Exsultet and opened my eyes to the beauty of the Gregorian Chant
\end{paragraph}

%%
% Table of Contents
%%

\tableofcontents
\clearpage

\mainmatter

\chapter{Introduction}
\clearpage
% The title:
\begin{center}\begin{huge}
\textsc{Allelia in Honor of Fr. Simon Le Moyne}
\end{huge}\end{center}

\section{Compiler Notice}
\begin{paragraph}
This collection is compiled and reprinted by Ben Sung Hsu
<pobetiger(at)gmail.com> for the express purpose of use at 
Mass at Sacred Heart of Jesus Chinese Parish. It contains a collection
of music from various sources. Unless otherwise noted, it contains
the Gospel Acclamation published by Corpus Christi Watershed \\
(\texttt{www.ccwatershed.org}) in honor of Fr. Simon Le Moyne. 
The Gospel Acclamation for Lent is No. 6 published by the same.
The Alleluia for Easter Triduum Vigil is the version found in the
Roman Gradual in the Graduale Triplex published by Solemes Abby.
\end{paragraph}

\section{Generic Formula}

\begin{paragraph}
This first section records the formula for any given verse. 
Subsequent verses used for each Sunday is then provided by section.
\end{paragraph}

% We type a text in the top right corner of the score:
\commentary{{\small \emph{Corpus Christi Watershed}}}
% Top-Level Alleluia:
\includescore{alleluia.autogen-tex}

% Generic Verse:
\begin{paragraph}\noindent\begin{large}
Generic Verse:\\\\
\end{large}\end{paragraph}
\gresetfirstlineaboveinitial{\small \textsc{\textbf{V/.}}}{\small \textsc{\textbf{V/.}}}
\includescore{Verse-Generic.autogen-tex}

\section{Video Instructions}
\begin{paragraph}
You find video instructions on YouTube at the following URL:
\end{paragraph}

\chapter{ADVENT}

% Top-Level Alleluia:
\includescore{alleluia.autogen-tex}



\chapter{CHRISTMAS}

% Top-Level Alleluia:
\includescore{alleluia.autogen-tex}


\chapter{LENT}
\section{Notes about Lent}
\begin{paragraph}
During the season of Lent, the Gospel Acclamation of Alleluia is
not to be sung. Another acclamation is used.
\end{paragraph}
% need to use something else here



\chapter{HOLY WEEK}

% Top-Level Alleluia:
\includescore{alleluia.autogen-tex}


\chapter{\textsc{Sacred Triduum of the Pssion and Resurrection of Our Lord}}
%{Passion and Resurrection of Our Lord}

\section{Holy Thursday}
\section{Easter Saturday, Vigil}
\section{Easter Sunday, Dawn}
\section{Easter Sunday, Day}
% Top-Level Alleluia:
\includescore{alleluia.autogen-tex}


\chapter{PASCHAL TIDE}

% Top-Level Alleluia:
\includescore{alleluia.autogen-tex}

% Verses Propers for each Sunday:

% Easter-Sunday-6
\begin{paragraph}\noindent\begin{large}
Verse for 6th Sunday in Easter (ABC):
\end{large}\newline\end{paragraph}

\begin{paragraph}\noindent
Not complete\newline
\end{paragraph}

\includescore{Easter-6.autogen-tex}

\chapter{ORDINARY TIME}

\includescore{alleluia.autogen-tex}


% OT-Sunday-27
\begin{paragraph}\noindent\begin{large}
Verse for 27th Sunday in Ordinary Time (B):
\end{large}\newline\end{paragraph}

\includescore{OT-Sunday-27.autogen-tex}

% OT-Sunday-28
\begin{paragraph}\noindent\begin{large}
Verse for 28th Sunday in Ordinary Time (B):
\end{large}\newline\end{paragraph}

\includescore{OT-Sunday-28.autogen-tex}

% OT-Sunday-29
\begin{paragraph}\noindent\begin{large}
Verse for 29th Sunday in Ordinary Time (B):
\end{large}\newline\end{paragraph}

\includescore{OT-Sunday-29.autogen-tex}

% OT-Sunday-30
\begin{paragraph}\noindent\begin{large}
Verse for 30th Sunday in Ordinary Time (B):
\end{large}\newline\end{paragraph}

\includescore{OT-Sunday-30.autogen-tex}


\end{document}
