% !TEX TS-program = lualatex
% !TEX encoding = UTF-8

% This is a simple template for a LuaLaTeX document using gregorio scores.

%\documentclass[11pt]{article} % use larger type; default would be 10pt
\documentclass[16pt,oneside,notitlepage]{article} % use larger type; default would be 10pt

% usual packages loading:
\usepackage{luatextra}
\usepackage{graphicx} % support the \includegraphics command and options
\usepackage{geometry} % See geometry.pdf to learn the layout options. There are lots.
%\geometry{a4paper} % or letterpaper (US) or a5paper or....
\geometry{letterpaper}
\usepackage{gregoriotex} % for gregorio score inclusion
\usepackage{fullpage} % to reduce the margins

% choose the language of the document here
\usepackage[latin]{babel}

% use the two following package for using normal TeX fonts
\usepackage[T1]{fontenc}
\usepackage[utf8]{luainputenc}
\usepackage{multicol}

% If you use usual TeX fonts, here is a starting point:
%\usepackage{times}
\usepackage{newcent}
%\usepackage{palatino}
%\usepackage{pslatex}

% to change the font to something better, you can install the kpfonts package (if not already installed). To do so
% go open the "TeX Live Manager" in the Menu Start->All Programs->TeX Live 2010

% here we begin the document
\begin{document}

\title{Introduction to Sacred Music}
\author{Ben Hsu}
\date{\today}
\maketitle

\begin{abstract}
This section contains the abstract.
\end{abstract}

% Here we set the space around the initial.
% Please report to http://home.gna.org/gregorio/gregoriotex/details for more details and options
\setspaceafterinitial{2.2mm plus 0em minus 0em}
\setspacebeforeinitial{2.2mm plus 0em minus 0em}

% Here we set the initial font. Change 43 if you want a bigger initial.
\def\greinitialformat#1{%
  {\fontsize{43}{43}\selectfont #1}%
}

% \addtolength{\topmargin}{-.50in}

%% \setgrefactor{12}
\pagenumbering{gobble}

% We set red lines here, comment it if you want black ones.
\redlines

\section{Header}

\begin{paragraph}
This is a test
\end{paragraph}

%% \begin{multicols}{2}
%% \includescore{gloria-viii.autogen-tex}
%% \columnbreak
%% \includescore{sanctus-xviii.autogen-tex}
%% \begin{paragraph}
%% \newline
%% \end{paragraph}
%% \includescore{agnus-dei-xviii.autogen-tex}
%% \end{multicols}

\end{document}

